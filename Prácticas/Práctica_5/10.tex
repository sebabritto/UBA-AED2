\section{Ejercicio 10: casiSort}

\subsection{}
\begin{algorithm}[H]
\caption{
    \textbf{ordenarConCasiSort}(\textbf{in/out} A: arreglo(nat))
}
    \begin{algorithmic}[1]
        \If{tam(A) == 2}
            \If{A[1] > A[2]}
                \State Swap(A[1], A[2])
            \EndIf
        \Else
            \State casiSort(A) \Comment{O(n)}
            \State mitadIzq $\gets$ A[1...tam(A)/2] \Comment{O(n)}
            \State mitadDer $\gets$ ordenarConCasiSort([tam(A)/2+1...tam(A)]) \Comment{O(n)}
            \State A $\gets$ merge(mitadIzq, mitadDer) \Comment{O(n)}
        \EndIf
    \end{algorithmic}
    \Complexity{$O(n)$}
\end{algorithm}

\subsection{}
La complejidad es O(n)

\subsection{}
No, no es posible sin tener mas informacion, si tuviera una cota podria usar BucketSort, CountingSort o RadixSort. Pero al no ser asi el caso la minima complejidad que puedo tener es O(nlog(n)), usando MergeSort, HeapSort, entre otros.