\section{Ejercicio 9: Planilla}
\subsection{}

\begin{algorithm}[H]
\caption{
    \textbf{ordenaPlanilla}(\textbf{in/out} p: planilla)
}
    \begin{algorithmic}[1]
    \State alumnos $\gets$ Vector::Vacia()  \Comment{$O(1)$}
    \State n $\gets$ Longitud(p)    \Comment{$O(1)$}
    \For{i $\gets$ 1 \textbf{to} 10}
        \For{j $\gets$ 1 \textbf{to} n} \Comment{$O(n)$}
            \If{p[j].puntaje = i}
                \State AgregarAtras(alumnos, p[j])
            \EndIf
        \EndFor
    \EndFor
    \State t $\gets$ 0
    \For{alum \textbf{in} alumnos} \Comment{$O(n)$}
        \If{alum.genero = FEM}
            \State p[t] = alum
            \State t $\gets$ t + 1
        \EndIf
    \EndFor
    \For{alum \textbf{in} alumnos} \Comment{$O(n)$}
        \If{alum.genero = MASC}
            \State p[t] = alum
            \State t $\gets$ t + 1
        \EndIf
    \EndFor
    \end{algorithmic}
    \Complexity{$O(n)$}
\end{algorithm}

\subsection{}
\begin{algorithm}[H]
\caption{
    \textbf{ordenaPlanilla}(\textbf{in/out} p: planilla)
}
    \begin{algorithmic}[1]
    \State alumnos $\gets$ Vector::Vacia()  \Comment{$O(1)$}
    \State n $\gets$ Longitud(p)    \Comment{$O(1)$}
    \For{i $\gets$ 1 \textbf{to} 10}
        \For{j $\gets$ 1 \textbf{to} n} \Comment{$O(n)$}
            \If{p[j].puntaje = i}
                \State AgregarAtras(alumnos, p[j])
            \EndIf
        \EndFor
    \EndFor
    \State t $\gets$ 0
    \For{g \textbf{in} GEN}
        \For{alum \textbf{in} alumnos} \Comment{$O(n)$}
            \If{alum.genero = g}
                \State p[t] = alum
                \State t $\gets$ t + 1
            \EndIf
        \EndFor
    \EndFor
    \end{algorithmic}
    \Complexity{$O(n)$}
\end{algorithm}

\subsection{}

El lower bound solo se aplica a algoritmos de ordenamiento basados en comparaciones. Y en nuestro algoritmo nunca comparamos 1 a 1 los elementos de nuestro arreglo.