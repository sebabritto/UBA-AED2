\section{Ejercicio 6: Escalera}

\begin{algorithm}[H]
\caption{
    \textbf{ordenarEscalera}(\textbf{in/out} r: recibe)
}
    \begin{algorithmic}[1]
        \State B $\gets$ obtenerEscaleras(A) \Comment{O(n)}
        \State mergeSort(B, A) \Comment{O(n log(n)), clave de ordenamiento: B[i][0] en el arreglo A.}
        \State mergeSortPorLong(B) \Comment{O(n log(n)), clave de ordenamiento: B[i][1] - B[i][0] + 1, es decir la longitud.}
        \State C $\gets$ Copiar(A)
        \State k $\gets$ 1
        \For{i $\gets$ 1 \textbf{to} Longitud(B)}
            \For{j $\gets$ B[i][0] \textbf{to} B[i][1]}
                \State A[k] $\gets$ C[j]
                \State k $\gets$ k + 1
            \EndFor
        \EndFor
    \end{algorithmic}
    \Complexity{$O(n log(n))$}
\end{algorithm}

\begin{algorithm}[H]
\caption{
    \textbf{obtenerEscaleras}(\textbf{in} A: arreglo(nat)) $\to$ \textbf{out} res: arreglo($\langle$ nat, nat $\rangle$) 
}
    \begin{algorithmic}[1]
        \State res $\gets$ Vector::Vacia()
        \State i $\gets$ 1
        \While{i $\leq$ n}
             \State j $\gets$ i
            \While{j $<$ n \yLuego A[j] + 1 = A[j+1]}
                \State j $\gets$ j + 1
            \EndWhile
                \State AgregarAtras(res, $\langle$i, j$\rangle$)
                \State i $\gets$ j + 1
        \EndWhile
    \end{algorithmic}
    \Complexity{$O(n)$}
\end{algorithm}
